\documentclass[11pt,a4paper]{article}
\usepackage[utf8]{inputenc}
\usepackage[english]{babel}
\usepackage{amsmath,amsfonts,amssymb,amsthm}
\usepackage{geometry}
\usepackage{graphicx}
\usepackage{hyperref}
\usepackage{enumerate}
\usepackage{fancyhdr}

\geometry{margin=2.5cm}
\pagestyle{fancy}
\fancyhf{}
\rhead{FTML 2025 Project}
\lhead{Fundamentals of Machine Learning Theory}
\cfoot{\thepage}

% Theorem environments
\newtheorem{theorem}{Theorem}[section]
\newtheorem{lemma}[theorem]{Lemma}
\newtheorem{proposition}[theorem]{Proposition}
\newtheorem{corollary}[theorem]{Corollary}
\theoremstyle{definition}
\newtheorem{definition}[theorem]{Definition}
\newtheorem{example}[theorem]{Example}
\theoremstyle{remark}
\newtheorem{remark}[theorem]{Remark}

\title{\textbf{Foundations of Machine Learning Theory}\\
\large Mathematical Analysis of Bayes Optimal Prediction, Loss Functions, and OLS}
\author{FTML 2025 Project}
\date{\today}

\begin{document}

\section{Bayes Risk with Absolute Loss}
\subsection{Question 0}

\begin{enumerate}
  \item Let \( f : \mathbb{R} \to \mathbb{R} \) such that
  \[
    f(x) = x^3
  \]
  \[
    f'(x) = 3x^2, \quad f'(0) = 0, \quad f''(x) = 6x, \quad f''(0) = 0
  \]
  \[
    For x < 0: f(x) = x³ < 0, For x > 0: f(x) = x³ > 0
  \]
  Therefore, f is strictly increasing around x = 0, making it an inflection point rather than a
local extremum
\end{enumerate}

\subsection{Question 1}

\begin{enumerate}
  \item Let \(Y\) be a random variable with
  \[
    P(Y=0) = 0.1, \quad P(Y=1) = 0.9
  \]
  \[
    f^*_{\text{squared}} = \mathbb{E}[Y] = 0.1 \times 0 + 0.9 \times 1 = 0.9
  \]

  \item Set
  \[
    h_1 = 0.9 \quad (\text{conditional mean})
  \]
  \[
    h_2 = 1 \quad (\text{empirical median})
  \]

  \item We want to calculate the absolute risk \( R_{\text{absolute}}(h) = \mathbb{E}[|Y - h|] \)

  \[
    R_{\text{absolute}}(h_1) = 0.1 \times |0 - 0.9| + 0.9 \times |1 - 0.9| = 0.09 + 0.09 = 0.18 = R_{\text{absolute}}(f^*_{\text{squared}})
  \]

  \[
    R_{\text{absolute}}(h_2) = 0.1 \times |0 - 1| + 0.9 \times |1 - 1| = 0.1
  \]

  \item Conclusion
  \[
    R_{\text{absolute}}(h_2) < R_{\text{absolute}}(f^*_{\text{squared}}) = 0.18
  \]

  \item The estimator \( h \) is 1

  \item
  \[
    f^*_{\text{absolute}}(x) = \arg \min_{z \in \mathbb{R}} g(z)
  \]
\end{enumerate}

\subsection{Question 2}
\begin{align*}
&\text{We want to show that the minimizer of } g(z) \text{ is the median of the distribution } Y|X=x. \\[6pt]
& g(z) \text{ is a convex function: it is a weighted average of convex functions } |y-z| \quad (\text{see part 2.3.1}) \\[6pt]
&\Rightarrow \text{every local minimum is a global minimum} \\[6pt]
&\text{So we can find the minimum of } g(z) \\[6pt]
& g(z) = \int_{-\infty}^{+\infty} |y-z|\, p_{Y|X=x}(y) \, dy \\[6pt]
&\frac{d}{dz} |y-z| = \begin{cases}
-1 & y > z \\
1 & y < z \\
0 & y = z
\end{cases} \quad \text{with} \quad |y-z| = \begin{cases}
y - z & y > z, \quad (y-z)' = -1 \\
z - y & z > y, \quad (z-y)' = 1
\end{cases} \\[12pt]
&\Rightarrow \delta'(z) = \int_{y < z} p(y) \, dy - \int_{y > z} p(y) \, dy = P(Y < z | X = x) - P(Y > z | X = x) \\[6pt]
&= P(Y < z | X = x) - 1 + P(Y < z | X = x) = 2 P(Y < z | X = x) - 1 \\[6pt]
&= 2 F(z) - 1 \\[6pt]
&\delta'(z^*) = 0 \\[6pt]
&\Rightarrow 2 F(z^*) - 1 = 0 \\[6pt]
&\Rightarrow F(z^*) = \frac{1}{2} = 0.5 \\[6pt]
&\Rightarrow z^* \text{ is the median of } Y|X=x \\[6pt]
&\delta''(z) = 2 p(Y = z | X = x) > 0 \Rightarrow z^* \text{ is indeed a local minimum} \\[6pt]
&\Rightarrow f_{\text{absolute}}^*(x) = \arg \min_{z \in \mathbb{R}} g(z) = z = \text{median}(Y|X=x)
\end{align*}


\end{document}
